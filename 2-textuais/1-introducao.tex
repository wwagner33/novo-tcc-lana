\chapter{Introdução}
\label{cap:introducao}

%Para começar a usar este \textit{template}, na plataforma \textit{ShareLatex}, vá nas opções (três barras vermelhas horizontais) no canto esquerdo superior da tela e clique em "Copiar Projeto" e dê um novo nome para o projeto. 

Em um cenário empresarial dinâmico e altamente competitivo, a eficácia na gestão de processos e a busca contínua por aprimoramento são essenciais para a sobrevivência e o sucesso de qualquer organização. A qualidade dos produtos e serviços fornecidos aos clientes desempenha um papel crítico neste contexto. Surge, então, o Planejamento Avançado da Qualidade do Produto como uma metodologia estratégica essencial. Esta abordagem visa assegurar a qualidade, eficácia e conformidade de produtos e serviços, integrando princípios de gestão da qualidade a estratégias de gestão de projetos.

No Grupo Industrial de Ações Automotivas, o Planejamento Avançado da Qualidade do Produto representa um conjunto estruturado de práticas e diretrizes que abarcam o planejamento, desenvolvimento e lançamento de produtos e serviços, conforme delineado pela \textit{Automotive Industry Action Group} em 2008\cite{B0082OP29K:2009}. Historicamente, a implantação dessa metodologia envolvia um grande volume de documentos, planilhas e processos manuais sujeitos a erros, dificultando a colaboração em tempo real entre equipes interdisciplinares. O desafio presente é aprimorar essa metodologia através de tecnologias modernas, tornando-a mais eficiente, precisa e acessível.

A aplicação desktop online desenvolvida para o gerenciamento eficiente do Planejamento Avançado da Qualidade do Produto oferece potenciais benefícios significativos para as organizações que a implementam, incluindo melhoria na eficiência, precisão e consistência, tomada de decisão informada, redução de custos e aumento da satisfação do cliente. Estes benefícios otimizam a gestão da qualidade e fortalecem a competitividade das organizações em um mercado exigente e dinâmico.

Desenvolver um sistema computacional para o Planejamento Avançado da Qualidade do Produto é um desafio empolgante, apresentando oportunidades para melhorar a gestão da qualidade e a eficiência dos processos em organizações que dependem deste sistema. Os principais desafios incluem a definição clara e compreensível dos requisitos, segurança de dados, usabilidade e design de interface do usuário, adoção e treinamento dos funcionários, manutenção e atualização contínua do sistema, e conformidade com regulamentações específicas do setor.

A superação desses desafios pode resultar em melhorias significativas na qualidade, eficiência e competitividade da organização. Este esforço está alinhado com o objetivo deste trabalho: Conduzir um estudo de caso no desenvolvimento de um software de gestão de qualidade para a indústria automobilística, com foco especial no Design Centrado no Usuário e na aplicação de metodologias de prototipagem digital, visando aprimorar a usabilidade e eficiência da interface do usuário.

\section{Objetivo Geral}

Conduzir um estudo de caso no desenvolvimento de um software de gestão de qualidade para a indústria automobilística, com foco especial no Design Centrado no Usuário e na aplicação de metodologias de prototipagem digital, visando aprimorar a usabilidade e eficiência da interface do usuário.

\section{Objetivos Específicos}

Para alcançar o objetivo geral, este estudo se propõe a abordar os seguintes pontos:

\begin{enumerate}
    \item Investigar os conceitos e práticas do Planejamento Avançado da Qualidade do Produto.

    \item Desenvolver e avaliar uma Interface de Usuário para Software de Gestão de Qualidade.

    \item Conduzir um Estudo de Caso iterativo com \textit{feedback} dos Usuários.

\end{enumerate}

%Testando o símbolo $\symE$

%\lipsum[5]  % Simulador de texto, ou seja, é um gerador de lero-lero.

%	\begin{alineas}
%		\item Lorem ipsum dolor sit amet, consectetur adipiscing elit. Nunc dictum sed tortor nec viverra.
%		\item Praesent vitae nulla varius, pulvinar quam at, dapibus nisi. Aenean in commodo tellus. Mauris molestie est sed justo malesuada, quis feugiat tellus venenatis.
%		\item Praesent quis erat eleifend, lacinia turpis in, tristique tellus. Nunc dictum sed tortor nec viverra.
%		\item Mauris facilisis odio eu ornare tempor. Nunc dictum sed tortor nec viverra.
%		\item Curabitur convallis odio at eros consequat pretium.
%	\end{alineas}



