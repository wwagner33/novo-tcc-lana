
Ao Prof. Me. Wellington Wagner Ferreira Sarmento por me orientar em meu trabalho de conclusão de curso e acreditar em todo o meu potencial.

Ao Prof. Dr. Tobias Rafael Fernandes Neto, coordenador do Laboratório de Sistemas Motrizes (LAMOTRIZ) onde este \textit{template} foi desenvolvido. 

Ao Doutorando em Engenharia Elétrica, Ednardo Moreira Rodrigues, e seu assistente, Alan Batista de Oliveira, aluno de graduação em Engenharia Elétrica, pela adequação do \textit{template} utilizado neste trabalho para que o mesmo ficasse de acordo com as normas da biblioteca da Universidade Federal do Ceará (UFC). %AVISO: Você pode usar este template uma vez que der os devidos créditos. Portanto, mantenha este parágrafo de agradecimento.

Aos meus pais, Celi e Ubaldo que acompanharam todo o processo de estudar trabalhar e crescer ao mesmo tempo. Lidando com todas as minhas preocupações, grandes e pequenas.

A minha irmã Gabrielly, que me ajudou a entender que o presente é feito a partir da constante dedicação, obrigada por nunca desistir de mim.

Aos queridos amigos, Loana Russo, Mário Valney, Humberto Lopez, Neto Costa, João Eduardo, Bruna Ribeiro, Pedro Oliveira, Eli Rodrigo, Jacó Farias, Alexandre Saraiva, Marcelle Pimentel, Daniel Andrade,  Bruno Esteves, e tantos outros, sem vocês essa trajetória não seria possível

Agradeço a todos os professores, por me proporcionar o conhecimento não apenas racional, mas a manifestação do caráter e afetividade da educação no processo de formação profissional,. Em especial às Profas. Dra. Cátia Luzia, Dra. Ticianne Darin, Dra.Adriana Holtz Betiol sem as quais eu não teria chegado tão longe como designer.

Aos colegas de trabalho da empresas Avicena, WDA, Altis Lab, Gringo, e e-Lastic pelo apoio e espaço para colocar todo o conhecimento teórico em prática.

Agradecimento aos maravilhosos Nicholas Trece  e Mayke Bessa, pelas noites de companhia, escrita e conversas em todos os momentos possíveis.