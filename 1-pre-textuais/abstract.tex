This research delves into the Advanced Product Quality Planning (APQP) system, recognized as a pivotal and structured process within the industry for the development of products and high-quality production processes. APQP encompasses the meticulous definition of customer requirements, comprehensive risk analysis, formulation of control plans, and rigorous validations of processes and products, all directed towards ensuring that the final product not only meets but exceeds customer expectations while being manufactured efficiently and consistently with high-quality standards.

The outlined objectives include the proposition of automating and streamlining the APQP process through the introduction of a computational system. This system is designed to centralize data management and foster seamless collaboration among teams, encompassing facets such as the definition of customer requirements, in-depth risk analysis, formulation of control plans, document management, generation of progress reports, and the maintenance of an activity history.

This research identifies formidable challenges, including the imperative of involving all stakeholders, the effective management of substantial volumes of data, adaptability to perpetually evolving requirements, and the requisition of additional resources such as time, personnel, and specialized software tools. The efficacy of APQP is contingent upon collaborative efforts across diverse organizational domains, thereby underscoring the criticality of surmounting these challenges to ensure the sustained effectiveness of the process.

By addressing the implementation of the computational system as a strategic solution to these challenges, this undergraduate thesis significantly contributes to the scholarly discourse and ongoing refinement of the APQP process within the industry. It aspires to enhance operational efficiency, flexibility, and overall quality in the sphere of product development.


% Separe as Keywords por ponto
\keywords{Design. APQP. Automotive Industry. User exeriencie. User interface.}