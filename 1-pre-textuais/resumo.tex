Este trabalho aborda o sistema Advanced Product Quality Planning (APQP) como um processo estruturado crucial na indústria para desenvolver produtos e processos de produção de alta qualidade. O APQP engloba a definição de requisitos do cliente, análises de risco, planos de controle e validações de processo e produto, visando garantir que o produto final atenda às expectativas do cliente e seja fabricado eficientemente com qualidade consistente.

Os objetivos do aqui apresentados incluem a proposta de automação e agilização do processo APQP por meio da implementação de um sistema computacional. Este sistema visa centralizar o gerenciamento de dados e facilitar a colaboração entre equipes, abrangendo a definição de requisitos do cliente, análises de risco, criação de planos de controle, gerenciamento de documentos, relatórios de progresso e histórico de atividades. 

O trabalho identifica desafios significativos, como o envolvimento de todas as partes interessadas, o gerenciamento eficaz de grandes volumes de dados, a adaptação a requisitos em constante mudança e a necessidade de recursos adicionais, como tempo, pessoal e ferramentas de software. O êxito do APQP depende da colaboração entre diversas áreas, tornando crucial a superação desses desafios para garantir a eficácia do processo.

Ao abordar a implementação do sistema computacional como uma solução para esses desafios, o traballho contribui para a compreensão e melhoria contínua do processo APQP na indústria, buscando eficiência, flexibilidade e qualidade aprimorada no desenvolvimento de produtos.


% Separe as palavras-chave por ponto
\palavraschave{Design. APQP. Industria automobilística. User exeriencie. User interface.}